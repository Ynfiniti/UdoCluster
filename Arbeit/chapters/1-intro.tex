\chapter{Einleitung}
\label{chap:1-intro}
\pagenumbering{arabic}
\setcounter{page}{1}

Dieses Kapitel beschreibt die zentralen Punkte der Motivation und der Aufgabenstellung. Dabei werden die Gründe der Aufgabe sowie die Vorgehensweise für die Entwicklung des Produkts erläutert.

\section{Motivation} 
\label{sec:motivation}
Die New York Times (NYT) ist eine eine US-amerikanische Tageszeitung\autocite{NewYorkTimes2023}
Die 

\section{Aufgabenstellung} 
\label{sec:problem}


\\
Die Labore der DHBW Mannheim sind das Zuhause verschiedenster Geräte, Experimente oder Projekte. In den Laboren finden sich neben technischen Geräten für Vorlesungen auch Ergebnisse von Studienarbeiten und anderen 
studentischen Ausarbeitungen. Bei einer solchen Menge von Geräten aus verschiedenen Quellen und für verschiedene Zwecke gestaltet sich die Übersicht sowie die Verwaltung schwierig. Interessiert sich eine Person für 
ein Gerät, so müssen die notwendigen Dokumente zuerst herausgesucht werden. Das führt zu einer großen Menge an Papier und Informationen verschiedener Medientypen an verschiedenen Orten, welche nicht zentral verwaltet 
werden. Um schnell zugängliche Informationen über verschiedene Geräte der Labore anbieten zu können, sollen an diesen QR-Codes angebracht werden, welche mit einem Mobilgerät einlesbar sind. Ein solcher QR-Code führt 
zu einer Informationsseite, auf welcher sämtliche Informationen für dieses Gerät gesammelt sind. So könnte auf dieser Seite Text geschrieben oder Bilder, Videos und Dokumente hochgeladen sein. Dadurch würde durch das 
Mobilgerät das schnelle Laden von sämtlichen relevanten Informationen ermöglicht werden. Dafür sollen verschiedene Technologien zum Entwurf eines Web-Servers evaluiert werden, der die Verwaltung der Informationen ermöglicht 
und die QR-Codes anbietet. Ebenso sollen nötige Prozesse definiert und geplant werden, sodass eine Implementierung gezielt durchgeführt werden kann. 

Ziel der Arbeit ist eine funktionsfähige Web-Anwendung zur Verwaltung von Informationen verschiedener Medientypen. Diese stellt Informationsseiten bereit, die durch das Einscannen von QR-Codes aufgerufen werden. Dadurch 
können Studenten oder andere Nutzer QR-Codes an Geräten einscannen und sämtliche relevante Informationen zu diesen einsehen können.

\section{Vorgehensweise} 

